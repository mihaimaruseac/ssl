\documentclass[a4paper,12pt]{article}
\usepackage[utf8x]{inputenc}
\usepackage{ucs}
\usepackage[english,romanian]{babel}
\usepackage{amsfonts}
\usepackage{amssymb}
\usepackage{amsmath}
\usepackage{hyperref}

\title{Solving CAPTCHAs}
\author{Mihai Maruseac, Lucian Mogoșanu, Sofia Neață, Adrian Șendroiu}
\date{March 2012}

% TODO: use a different template/layout for paper/journal
% TODO: restructure tex if document has the change of getting big
\begin{document}

\maketitle

\subsection*{Project motivation}

The CAPTCHA (Completely Automated Public Turing test to tell
Computers and Humans Apart) mechanism is a measure widely used on
the World Wide Web to provide tests solvable only by human users.
It is commonly used as a means to prevent problems such as automated
spam, website registration, collection of e-mail addresses or abuse
of services such as online voting.

Usually CAPTCHAs consist of a challenge-response test in which the
user is provided with an image containing a small number of words
and is asked to type them back. This kind of test is in theory
easily solvable by a human agent, while the problem of recognizing the
text in an image is not trivial from the point of view of algorithms.
Moreover, the CAPTCHA usually contains noise and/or distorted text,
making the use techniques such as Optical Character Recognition
difficult. This is similar to the manner in which one-way functions
are used in cryptographic algorithms.

Our project aims to use Machine Learning for the purpose of solving 
CAPTCHAs. There are several approaches to this. One approach would
involve using a free/open source CAPTCHA generator to generate
a set of training and/or test examples and try to improve on
various algorithms known to work well on this problem. Testing
against well-known CAPTCHA systems such as Google's reCAPTCHA
\footnote{\url{http://www.google.com/recaptcha}} could also give
us a good idea about the performance of our algorithm(s), as
well as the effectiveness of CAPTCHA as a security mechanism.

\end{document}
